\chapter{Document Classes}
\label{cha:document-classes}

To this point, we have been talking pretty much about how to get
\LaTeX{} to make text look the way you want, and we've always included
the line \verb|\documentclass{article}|.  This is all well and good,
and we get what we want, but you might be wondering: why only
\texttt{article}?  Does \LaTeX{} do other things?  And what's an
article anyway?

\LaTeX{} has several built-in document classes: \texttt{article},
\texttt{report}, \texttt{book}, \texttt{slides}, and \texttt{letter},
each of which are used pretty much for precisely what the name
implies.  \texttt{article} is also used, as you have learned, in more
general situations as the generic document class.  These are not on
the only document classes in existence, as individuals and
organizations can write their own, but they are the standard ones.

\section{Standard \LaTeX{} Document Classes}
\label{sec:stand-latex-class}

\subsection{Article}
\label{sec:article}

\texttt{article} is used for, as the name implies, journal articles.
It is the most basic \LaTeX{} document class, as it has few special
commands and quirks.  (This may be less because \texttt{article} is
somehow inherently superior and more because it is the document class
everyone thinks of first.  But maybe that makes it inherently
superior.)  The title and associated information is centered by the
\verb|\maketitle| command.  If one wants to include an abstract, this
can be done with the \texttt{abstract} environment.

\subsection{Report}
\label{sec:report}

\texttt{report} is very similar to \texttt{article}.  However, it is
designed to be used for longer documents, and as such, has two
additional sectioning commands: \verb|\chapter| and \verb|\part|.
Additionally, as one might expect, the title information appears on a
title page rather than simply at the top of the first page.  The
abstract gets its own page as well.

\subsection{Slides}
\label{sec:slides}

\texttt{slides}, not very surprisingly, makes slides.  The slides are
meant to be printed and projected, rather than projected from the
computer as you might with a PowerPoint.  This isn't to say you can't
do it, but people have written other document classes to achieve this
end.  Additionally, \texttt{slides} documents are a bit trickier in
their construction than other types of documents.  In general, it is
best to stay away from \texttt{slides} unless you actually intend to
print them and project them on an overhead.  If you are interested in
creating computer presentations for use on screen projectors, use
Beamer, an add-on to \LaTeX{} that provides just such a functionality.
Continuing with \texttt{slides}, we're going to look at the general
format first and then make some more general comments.

\subsubsection{Format of a \texttt{slides} Document}
\label{sec:form-slides-docum}

\begin{verbatim}
\documentclass{slides}
preamble
\title{title}
\author{author}
\date{date}

\begin{document}
\maketitle

\begin{slide}
  slide content
\end{slide}

\end{document}
\end{verbatim}

The things that one should take from this skeleton of an example is
that the document class is called \texttt{slides}, not \texttt{slide}
(this is a seemingly small detail, but when your documents won't
compile because you forgot what the document class was called, that's
going to become an important point); the title information does not go
within a slide, it goes before the slides start; lastly, the contents
of each slide are set off within a \texttt{slide} environment.

There's some information you probably didn't glean from the little
skeleton (good job if you figured this out, because the information is
not in the example to be observed).  First, a fact that you may want
to know or just care about: \texttt{slides} uses a different default
font than the other \LaTeX{} document classes.  Additionally, it is
your responsibility to make sure that what you want to go on one slide
fits on one slide.  If the stuff between the \verb|\begin{slide}| and
  \verb|\end{slide}| is going to spill over onto another page,
\LaTeX{} lets it and simply doesn't number the overflow slide(s).

\texttt{slides} documents can have two other environments besides
\texttt{slide}: \texttt{overlay} and \texttt{note}.  \texttt{overlay}
is used for making a slide designed to be placed over the slide
environment preceding it.  As a result, overlays aren't numbered in
the same way as slides; the first overlay corresponding to the fourth
slide would be numbered 4-a.  In order to get the text on the overlay
to line up correctly, it is best to include the text from the slide
with the color set to white.  This can be accomplished with
\verb|\textcolor{white}{text}|.

The \texttt{note} environment allows one to make notes corresponding
to each slide.  Each note is numbered according to its corresponding
slide, so the first note corresponding to the fourth slide would be
numbered 4-1.

\subsection{Letters}
\label{sec:letters}

The \texttt{letter} document class has several commands specific to
it.  It bears some similarity to the \texttt{slides} class because you
can have multiple letters in one document.  The return address is
controlled by the \verb|\address| command in the preamble.  Each line
of the return address (and the other command arguments in letters, if
necessary) is separated by a $\backslash\backslash$.  The start of the
\texttt{letter} environment is structured as
\begin{verbatim}
\begin{letter}{addressee}
\end{verbatim}
where \emph{addressee} is both the name and address of the addressee.
Next comes the \verb|\opening{opening}| argument, which has as its
argument the opening of the letter.  The letter concludes with the
\verb|\closing{}| and \verb|\signature{}| commands where
\verb|\closing| takes the closing of the letter and
\verb|\signature| the name and/or title appearing under the
signature as their arguments.  The \verb|\cc| and \verb|\encl|
commands are optional and take the names of those receiving copies
and the list of enclosures.

\section{Nonstandard Document Classes}
\label{sec:nonst-docum-class}

There are many nonstandard document classes out there, but two of the
most popular ones are the KOMA-Script document class bundle and the
\texttt{memoir} document class.  They are meant to serve as
replacements to the default \LaTeX{} document classes, offering more
features, and in both cases, typographically better documents.
Besides those two, the \AmS{} provides its own document classes for
articles, books, and monographs that are submitted to various
mathematical journals.

\subsection{KOMA-Script\protect\footnote{\url{http://www.ctan.org/tex-archive/help/Catalogue/entries/koma-script.html}}}
\label{sec:koma-script}

The KOMA-Script bundle (written by Markus Kohm) provides drop-in
replacements for the \texttt{article}/\texttt{report}/\texttt{book}
classes with emphasis on typography and versatility.  There is also a
letter class, different from all other letter classes.  It also offers
a package for calculated type areas in the way laid down by the
typographer Jan Tschichold, a package for easily changing and defining
of page styles, a package for getting not only the current date, but
also the name of the day, and a package for getting the current time.
All these packages may be used not only with KOMA-Script classes, but
also with standard classes.

The equivalents to the \texttt{article}, \texttt{report},
\texttt{book}, and \texttt{letter} classes are \texttt{scrartcl},
\texttt{scrrprt}, \texttt{scrbook}, and \texttt{scrlttr2},
respectively.  By default, the paper size for all four classes is set
to A4 paper, so you have to manually change it to letter paper by
using
\begin{verbatim}
\documentclass[letterpaper]{foo}
\end{verbatim}
where \texttt{foo} is the name of the KOMA-Script document class you
want to use. Furthermore, the default font is set to 11pt.  Please see
the KOMA-Script documentation for more details on the classes.

\subsection{Memoir\protect\footnote{Taken from the README file in the
    CTAN directory
    (\url{http://www.ctan.org/tex-archive/macros/latex/contrib/memoir/})}}
\label{sec:memoir}

The \texttt{memoir} class is a flexible class for typesetting poetry,
fiction, non-fiction, and mathematical works as books, reports,
articles, or manuscripts.\footnote{This reader was typeset using the
  \texttt{memoir} class.}  It is meant to replace the standard
\LaTeX{} document classes, particularly the \texttt{report} and
\texttt{book} document classes.  Documents can use 9pt, 10pt, 11pt,
12pt, 14pt, or 17pt as the normal font size.  Many methods are
provided to let you create your particular design.  The class
incorporates over thirty of the more popular packages.  It was written
by Peter Wilson.  See the documentation for more information on the
wealth of customizations for the document class.

%%% Local Variables: 
%%% mode: latex
%%% TeX-master: "reader"
%%% End: 