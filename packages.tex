\chapter{Packages}
\label{cha:packages}

\section{Package Overview}
\label{sec:package-overview}

\LaTeX{} itself is a large package of macros written in the \TeX{}
language.  Accordingly, \LaTeX{} has been designed to facilitate the
creation and usage of user-defined packages to extend the
functionality of the \LaTeX{} system.  The large \LaTeX{}
distributions, such as \MiKTeX{}, ship with an enormous library of
packages and feature some kind of package managemetn software.  This
makes it easy to download, install, and manage packages on a \LaTeX{}
installation.  This chapter will summarize this and mention some of
the most common \LaTeX{} packages.  Much more descriptive lists of
packages are located in Appendix~\ref{cha:list-packages}.

\section{Package Management}
\label{sec:package-management}

The base of a \LaTeX{} installation (or any other flavor of \TeX{}) is
a directory called \verb|\texmf| by default.  Common locations for
this directory include \verb|/usr/share/texmf| on Mac OS X, Unix, and
Unix-like operating systems, and \verb|c:\texmf| on Windows operating
systems.  This directory will contain a collection of fonts, packages,
scripts, source code, etc.  The precise details will vary according to
the distribution.  \TeX{} was designed for deployment on large UNIX
systems, and the \texttt{texmf} tree is intended to be maintained by
the system administrator and is generally off-limits to users.  In the
days before \TeX{} distributions became common, an expert would have
to manually compile and configure the installation for use in a
network environment.  This remains the case in many multi-user
networks.  For personalized needs, every user has a
\texttt{localtexmf} directory in the home directory.  The contents of
the \texttt{localtexmf} tree override those of the \texttt{texmf}
tree.  This allows an individual user to load packages or fonts that
are not in the main \texttt{texmf} tree without disturbing the
delicate server-side installation and breaking \TeX{} for everyone on
the network.

However, the distinction between the main and local \TeX{} trees
is less clear for single-user operating systems or those with weak
multi-user support (i.e., Windows and Mac OS 9 and earlier).  In these
cases, the users and administrator are the same person.

The details of managing packages is different depending on the
distribution.  It is best to refer to the documentation that comes
with the \LaTeX{} distribution to see exactly how it is done.  The
process is fairly straightforward in \MiKTeX{} as it features a
graphical package management utility to automate the process.  The
package manager will connect to a CTAN mirror, download the package,
install it, and rehash the \TeX{} tree for you.

This can also be done manually.  The documentation for your
distribution or the package that you are installing should specify
details.  The process involves three generic steps.  First, you must
locate the package.  Nearly all mainstream packages reside on the CTAN
mirrors.  Once you have the package in hand, you must place it in the
desired location in your \TeX{} tree, typically an intuitively-named
directory in \texttt{localtexmf}.  Lastly, you need to rehash the
\TeX{} tree.  This rebuilds an index of all available packages that
\TeX{} maintains, allowing the computer to locate the package you want
when you request it from within a document.

\LaTeX{} packages are typically installed in \verb|\texmf\tex\latex|.
For organizational purposes, you may want to create a subdirectory for
the package.  After copying the package, you must rehash the \TeX{}
tree.  To save time, the \LaTeX{} compiler does not scan the \TeX{}
tree for packages when run.  Rather, it refers to a configuration file
to tell it which packages are installed, and where.  Typing
\texttt{texhash} from the command line should run the executable that
will refresh this file.  In Windows XP, the command line can be
accessed by selecting Start $\rightarrow$ Run, and entering
\texttt{cmd}.  On Mac OS X, the command line can be accessed from the
\texttt{terminal} application in the Utilities folder.  On any Unix or
Unix-like operating system, you should be knee-deep in ways to access
a command line.  It is important to note that to hash the \TeX{} tree,
you will need administrator privileges.  This is common on Windows and
Mac OS, but you will need to either be root or use a program such as
\texttt{sudo} on any Unix or Unix-like operating system.

If your computer cannot find the appropriate programs to hash your
\TeX{} tree (this will be somewhat more common on Windows than the
other operating systems), you may have to find where the binaries for
your \TeX{} installation are located to run \texttt{texhash}; refer to
your documentation for specifics.

\section{Packages of Interest}
\label{sec:packages-interest}

There are a ridiculous number of user-defined packages available on
the Internet.  A typical \LaTeX{} installation should include the most
common packages, as well as some that are not.  There are packages
available to simplify typesetting tasks in law, medicine, linguistics,
computer science, chemistry, music, and more, as well as mathematics.
Even if you have a very specific need, someone has probably written a
package to make it easy.  CTAN is the best and most organized source
of packages, and when it doubt, \href{http://www.google.com}{Google}
is your best friend.

\subsection{\AmS{} Packages}
\label{sec:ams-packages}

The American Mathematical Society (\AmS) maintains a family of
packages that are invaluable in typesetting mathematical discourse.
The \texttt{amsmath} package provides an expanded set of math symbols,
and should always be used when typesetting mathematics.  The
\texttt{amsfonts} package provides additional mathematical fonts and
should also be included in nearly all mathematical documents.  The
packages \texttt{amssymb} and \texttt{amsextra} are less essential,
but still good to include.

The \texttt{amsthm} package provides the \AmS{} theorem environments.
This allows for the insertion of \AmS-style theorem environments as
described in Section~\ref{sec:theorems-proofs}.

\subsection{Line Spacing}
\label{sec:line-spacing}

The \texttt{setspace} package adds the ability to doublespace
documents in \LaTeX{}.  To doublespace a document, put
\begin{verbatim}
\usepackage{setspace}
\end{verbatim}
and
\begin{verbatim}
\doublespace
\end{verbatim}
in the preamble.  Other line spacing options are available, and are
described in the documentation of the \texttt{setspace} package.

\subsection{Hyperref}
\label{sec:hyperref}

The \texttt{hyperref} package provides commands for typesetting
Internet URLs (using the \verb|\href| command).  This not only
improves the formatting of the URLs, but also adds hyperlinks to the
linked URL.  This is especially useful if you plan to distribute your
document as a PDF.  Here's the syntax for the \verb|\href| command:
\begin{verbatim}
\href{http://blah.org/blah.html}{name of link}
\end{verbatim}

In addition, by passing the option \texttt{colorlinks=true} to
\texttt{hyperref}, the rectangle box around links and internal
references will be replaced by changing the color of the text itself.
For internal references (especially in the table of contents and the
use of \verb|\ref|), the color is set to red; for hyperlinks set by
\verb|\href|, the default color is pink.

\subsection{Babel}
\label{sec:babel}

The \texttt{babel} package extends \LaTeX's language support.  Refer
to \texttt{babel}'s documentation for help with any specific language.

%%% Local Variables: 
%%% mode: latex
%%% TeX-master: "reader"
%%% End: 