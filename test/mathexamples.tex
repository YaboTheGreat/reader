\chapter{Math Examples}
\label{cha:math-examples}

The following is a collection of \LaTeX{} examples.  First, the source
code is displayed, immediately followed by the compiled code.  Try not
to pay attention to the mathematical content as it is culled from old
homework sets and may or may not be correct.

\section{Radicals and Fractions}
\label{sec:radicals-fractions}

This is an example of usages of radicals and fractions.  The
\texttt{amsmath} and \texttt{amssymb} packages are necessary.

\begin{verbatim}
\paragraph*{4.4}
Show that $1,\sqrt{2},\sqrt{3},\sqrt{6}$ are
linearly independent over $\mathbb{Q}$.
\subparagraph*{}
Suppose $1,\sqrt{2},\sqrt{3},\sqrt{6}$ are linearly
dependent over $\mathbb{Q}$. Then there exist
$p,q,r,s\in\mathbb{Q}$ such that
$p+q\sqrt{2}+r\sqrt{3}+s\sqrt{6}=0$ and
$p,q,r,s$ are not all zero. Then either $r$ or $s$
is nonzero, otherwise, $p=0$, or
$q\sqrt{2}\in\mathbb{Q}$, neither of which can be
the case if $p,q,r,s$ are not all zero. Then we have
\begin{displaymath}
  \sqrt{3}=\frac{-p-q\sqrt{2}}{r+s\sqrt{2}}
\end{displaymath}
and this is well defined because at least one of
$r$ and $s$ is not zero. Thus, there exist rational
numbers $e,f$ such that $\sqrt{3}=e+f\sqrt{2}$.
Squaring this, we have
\begin{align*}
  3 &= e^2+ef\sqrt{2}+2f^2 \\
  \frac{3-e^2-2f^2}{ef} &= \sqrt{2}.
\end{align*}
Which implies that $\sqrt{2}$ is rational.
Thus, we have a contradiction. Therefore, we can
conclude that $1,\sqrt{2},\sqrt{3},\sqrt{6}$ are
linearly independent over $\mathbb{Q}$.
\end{verbatim}

\paragraph*{4.4}
Show that $1,\sqrt{2},\sqrt{3},\sqrt{6}$ are linearly independent over
$\mathbb{Q}$.

\subparagraph*{}
Suppose $1,\sqrt{2},\sqrt{3},\sqrt{6}$
are linearly dependent over $\mathbb{Q}$. Then there exist
$p,q,r,s\in\mathbb{Q}$ such that $p+q\sqrt{2}+r\sqrt{3}+s\sqrt{6}=0$
and $p,q,r,s$ are not all zero. Then either $r$ or $s$ is nonzero,
otherwise, $p=0$, or $q\sqrt{2}\in\mathbb{Q}$, neither of which can be
the case if $p,q,r,s$ are not all zero. Then we have
\begin{displaymath}
  \sqrt{3}=\frac{-p-q\sqrt{2}}{r+s\sqrt{2}}
\end{displaymath}
and this is well defined because at least one of $r$ and $s$ is not
zero. Thus, there exist rational numbers $e,f$ such that
$\sqrt{3}=e+f\sqrt{2}$.  Squaring this, we have
\begin{align*}
  3 &= e^2+ef\sqrt{2}+2f^2 \\
  \frac{3-e^2-2f^2}{ef} &= \sqrt{2}.
\end{align*}
which implies that $\sqrt{2}$ is rational.  Thus, we have a
contradiction. Therefore, we can conclude that
$1,\sqrt{2},\sqrt{3},\sqrt{6}$ are linearly independent over
$\mathbb{Q}$.

Note that inline fractions can be made larger by using
\verb|\displaystyle|.

\section{Aligned Equations}
\label{sec:aligned-equations}

The following is an example of the \verb|\align*| environment.  The
\texttt{amsmath} package is necessary.

\begin{verbatim}
Conjugating the elements of $\mathbb{V}$ by the
transposition of $\mathbb{S}_4$, we have
\begin{align*}
  (12)(12)(34)(12) &= (12)(34) \\
  (12)(13)(24)(12) &= (14)(23) \\
  (12)(14)(23)(12) &= (13)(24) \\
  (13)(12)(34)(13) &= (14)(23) \\
  (13)(13)(24)(13) &= (13)(24) \\
  (13)(14)(23)(13) &= (12)(34) \\
  (14)(12)(34)(14) &= (13)(24) \\
  (14)(13)(24)(14) &= (12)(34) \\
  (14)(14)(23)(14) &= (14)(23) \\
  (23)(12)(34)(23) &= (13)(24) \\
  (23)(13)(24)(23) &= (12)(34) \\
  (23)(14)(23)(23) &= (14)(23) \\
  (24)(12)(34)(24) &= (14)(23) \\
  (24)(13)(24)(24) &= (13)(24) \\
  (24)(14)(23)(24) &= (12)(34) \\
  (34)(12)(34)(34) &= (12)(34) \\
  (34)(13)(24)(34) &= (14)(23) \\
  (34)(14)(23)(34) &= (13)(24)
\end{align*}
thus, $\mathbb{V}$ is closed under conjugation by
transpositions in $\mathbb{S}_4$ as $\mathbb{S}_4$
is generated by all transpositions on 4 letters,
we can conclude that $\mathbb{V} \lhd \mathbb{S}_4$.
\end{verbatim}

Conjugating the elements of $\mathbb{V}$ by the transposition of
$\mathbb{S}_4$, we have
\begin{align*}
  (12)(12)(34)(12) &= (12)(34) \\
  (12)(13)(24)(12) &= (14)(23) \\
  (12)(14)(23)(12) &= (13)(24) \\
  (13)(12)(34)(13) &= (14)(23) \\
  (13)(13)(24)(13) &= (13)(24) \\
  (13)(14)(23)(13) &= (12)(34) \\
  (14)(12)(34)(14) &= (13)(24) \\
  (14)(13)(24)(14) &= (12)(34) \\
  (14)(14)(23)(14) &= (14)(23) \\
  (23)(12)(34)(23) &= (13)(24) \\
  (23)(13)(24)(23) &= (12)(34) \\
  (23)(14)(23)(23) &= (14)(23) \\
  (24)(12)(34)(24) &= (14)(23) \\
  (24)(13)(24)(24) &= (13)(24) \\
  (24)(14)(23)(24) &= (12)(34) \\
  (34)(12)(34)(34) &= (12)(34) \\
  (34)(13)(24)(34) &= (14)(23) \\
  (34)(14)(23)(34) &= (13)(24)
\end{align*}
thus, $\mathbb{V}$ is closed under conjugation by transpositions in
$\mathbb{S}_4$ as $\mathbb{S}_4$ is generated by all transpositions on
4 letters, we can conclude that $\mathbb{V} \lhd \mathbb{S}_4$.

\section{Piecewise Functions}
\label{sec:piecewise-functions-1}

The following is an example of a piecewise function.

\begin{verbatim}
\paragraph*{3}
Let $m$ be a cardinal number such that $m+\aleph_0=C$.
Without the Axiom of Choice, show that $m=C$. Since
$m+\aleph_0=C$, we have $A\cup \omega \sim \lbrace
0,1 \rbrace^\omega$ where $A$ and $\omega$ are
disjoint. Let $f:\lbrace 0,1\rbrace^\omega \to A \cup
\omega$ be an equivalence. Consider the families of
functions $F_n$ and $G_n$ in $\lbrace 0,1\rbrace^
\omega$, for all $n\in \omega$ defined as follows:

\[
F_n(x) =
\begin{cases}
  0, & x \neq n; \\
  1, & x=n;
\end{cases}
\]
\[
G_n(x) =
\begin{cases}
  0, & x=n; \\
  1, & x \neq n;
\end{cases}
\]

Define a one-to-one and onto function
$g:\lbrace 0,1\rbrace^\omega \to \lbrace 0,1 %
\rbrace^\omega$ such that $f\circ g(F_n)\in
\omega$ for all $n\in \omega$ and $f\circ g(G_n)
\in A$ for all $n\in \omega$. Since $g$ is
one-to-one and onto, $f\circ g$ is an equivalence
from $\lbrace 0,1\rbrace^\omega$ to $A \cup \omega$.
Furthermore, since the set of all $F_n\in %
\lbrace 0,1\rbrace^\omega$ and the set of all
$G_n\in \lbrace 0,1\rbrace^\omega$ are both clearly
equivalent to $\omega$, we can conclude that there
is a subset $B$ of $A$ that is equivalent to $\omega$.
Thus, $A=(A-B) \cup B$. Hence, there is a cardinal
number $n$ such that $m=n+\aleph_0$. Therefore,
$m+\aleph_0=n+\aleph_0+\aleph_0=n+\aleph_0=m$. Thus,
$m=C$.
\end{verbatim}

\paragraph*{3}
Let $m$ be a cardinal number such that $m+\aleph_0=C$.  Without the
Axiom of Choice, show that $m=C$. Since $m+\aleph_0=C$, we have $A\cup
\omega \sim \lbrace 0,1 \rbrace^\omega$ where $A$ and $\omega$ are
disjoint.  Let $f:\lbrace 0,1\rbrace^\omega \to A \cup \omega$ be an
equivalence. Consider the families of functions $F_n$ and $G_n$ in
$\lbrace 0,1\rbrace^\omega$, for all $n\in \omega$ defined as follows:

\[
F_n(x) =
\begin{cases}
  0, & x \neq n; \\
  1, & x=n;
\end{cases}
\]
\[
G_n(x) =
\begin{cases}
  0, & x=n; \\
  1, & x \neq n;
\end{cases}
\]

Define a one-to-one and onto function $g:\lbrace 0,1\rbrace^\omega \to
\lbrace 0,1\rbrace^\omega$ such that $f\circ g(F_n)\in \omega$ for all
$n\in \omega$ and $f\circ g(G_n)\in A$ for all $n\in \omega$.  Since
$g$ is one-to-one and onto, $f\circ g$ is an equivalence from $\lbrace
0,1\rbrace^\omega$ to $A \cup \omega$.  Furthermore, since the set of
all $F_n\in \lbrace 0,1\rbrace^\omega$ and the set of all $G_n\in
\lbrace 0,1\rbrace^\omega$ are both clearly equivalent to $\omega$,
we can conclude that there is a subset $B$ of $A$ that is equivalent
to $\omega$.  Thus, $A=(A-B) \cup B$. Hence, there is a cardinal
number $n$ such that $m=n+\aleph_0$. Therefore,
$m+\aleph_0=n+\aleph_0+\aleph_0=n+\aleph_0=m$. Thus, $m=C$.

\section{XY}
\label{sec:xy}

The following is an example of drawing commutative diagrams with the
\texttt{xy} package.  For the following code to function, the package
\texttt{xy} must be included in the preamble of the document
(\verb|\usepackage[all]{xy}|).

\begin{verbatim}
Suppose $\phi:A\to B$ and $\psi:C\to D$ are
isomorphic field extensions. Then the following
diagram commutes.
\begin{displaymath}
\xymatrix{
A \ar[r]^\phi \ar[d]_\alpha & B \ar[d]_\beta \\
C \ar[r]^\psi & D}
\end{displaymath}
Thus, $\beta\circ\phi=\psi\circ\alpha$. Therefore
the diagram
\begin{displaymath}
\xymatrix{
C \ar[r]^\psi \ar[d]_\alpha^{-1} & D %
\ar[d]_\beta^{-1} \\
A \ar[r]^\phi & B}
\end{displaymath}
also commutes. Thus, $\psi$ is isomorphic to $\phi$.
Therefore, $\phi~\psi$ implies $\psi~\phi$.
\subparagraph*{}
Suppose $\phi:A\to B$ and $\psi:C\to D$ are
isomorphic field extensions. Furthermore, let
$\psi:C\to D$ and $\mu:E\to F$ be isomorphic field
extensions. The following diagrams commute.
\begin{displaymath}
\xymatrix{
A \ar[r]^\phi \ar[d]_\alpha & B \ar[d]_\beta \\
C \ar[r]^\psi & D}
\xymatrix{
C \ar[r]^\psi \ar[d]_\delta & D \ar[d]_\varepsilon \\
E \ar[r]^\mu & F}
\end{displaymath}
\end{verbatim}

Suppose $\phi:A\to B$ and $\psi:C\to D$ are isomorphic
field extensions. Then the following diagram commutes.
\begin{displaymath}
\xymatrix{
A \ar[r]^\phi \ar[d]_\alpha & B \ar[d]_\beta \\
C \ar[r]^\psi & D}
\end{displaymath}
Thus, $\beta\circ\phi=\psi\circ\alpha$. Therefore the diagram
\begin{displaymath}
\xymatrix{
C \ar[r]^\psi \ar[d]_\alpha^{-1} & D \ar[d]_\beta^{-1} \\
A \ar[r]^\phi & B}
\end{displaymath}
also commutes. Thus, $\psi$ is isomorphic to $\phi$.
Therefore, $\phi~\psi$ implies $\psi~\phi$.
\subparagraph*{}
Suppose $\phi:A\to B$ and $\psi:C\to D$ are isomorphic field
extensions. Furthermore, let $\psi:C\to D$ and $\mu:E\to F$
be isomorphic field extensions. The following diagrams commute.
\begin{displaymath}
  \xymatrix{
    A \ar[r]^\phi \ar[d]_\alpha & B \ar[d]_\beta \\
    C \ar[r]^\psi & D}
  \xymatrix{
    C \ar[r]^\psi \ar[d]_\delta & D \ar[d]_\varepsilon \\
    E \ar[r]^\mu & F}
\end{displaymath}

%%% Local Variables: 
%%% mode: latex
%%% TeX-master: "reader"
%%% End: 
