\chapter{An Overview of \LaTeX{}}
\label{cha:an-overview-latex}

\section{What It Does}
\label{sec:what-it-does}

\LaTeX{} dominates academic publishing in science and mathematics.  If
you want an academic career in these fields (particularly
mathematics), you must learn to love \LaTeX{}.  Those that merely want
to type up their homework can get by with a casual fondness.

While not as friendly as the cartoon paperclip, \LaTeX{} is not at all
difficult to use.  In fact, the great news is that \LaTeX{} possesses
a very flat learning curve.  This means that once you get into it,
getting at all the most advanced features is not that difficult.  A
little practice is all that is needed to produce textbook-quality
technical documents.  Unfortunately, for all of its virtues, \LaTeX{}
is not all that friendly to the beginner.  In fact, that small bit of
practice necessary for getting textbook-quality is not that far from
what is needed to do some basic typesetting in \LaTeX{}.

It will run on any modern computer system (and most unmodern computer
systems) and is well supported by a community of developers,
enthusiasts, and fanatical zealots.  It can be extended to satisfy a
wide variety of publishing needs, and a vast repository of such
user-made packages are available on the Internet.  Most significantly
for academia: \LaTeX{} is free.  Departments are too cheap to buy
licenses for big commercial software products, and students are too
lazy to learn how to use a pirated copy of Mathematica.  Before long,
you may decide to junk Microsoft Word and the stupid paperclip and do
all of your word processing with \LaTeX{}.

\section{Where It Came From}
\label{sec:where-it-came}

The \TeX{} language was written by Stanford computer scientist Donald
Knuth in the late 1970s.  Knuth is best known for \emph{The Art of
  Computer Programming} books which are of biblical proportions in
academic computer science.  Upset by the poor quality of the
typesetting in the published editions of his books, Knuth decided that
he could do better.  The first version of \TeX{} was completed in 1978
and ran on a huge mainframe computer at Stanford.  Although \TeX{}
began as a research project, the mathematics community became
interested because it promised to be a cheap solution for academic
publishing.  The American Mathematical Society even sponsored their
own \TeX{} implementation, predictably titled \AmS-\TeX{}.

In addition to \TeX{}, Knuth also wrote \MF{}, a language for
producing document fonts from mathematical curves.  This system, and
its successor \MP{}, are still used for generating fonts and figures
for technical documents.

The \TeX{} language itself is rather primitive.  You could typeset an
entire document in pure \TeX{} if you wanted to, but you could also
drive on the wrong side of the road.  Donald Knuth wrote what is
called Plain \TeX{}, a set of higher-level commands written in simpler
\TeX{} commands, to make things easier for mere mortals.  However,
even in Plain \TeX{}, the author has to do more work setting the type
than actually composing the document.  Various languages derived from
\TeX{} have been developed to be more functional than Plain \TeX{}.
Of these, \LaTeX{} is the most prevalent.  It was written in 1984 by
research computer scientist Leslie Lamport.  Thus, \LaTeX{} is not a
software package like Microsoft Office or even a stand-alone computer
language.  Rather, it is what is called a macro language\Dash{}a set of
commands written in \TeX{}.  In practice, it functions like a typical
markup language such as HTML\@.  Development on \LaTeX{} still
continues: the current version is called \LaTeXe{} and small updates
are made to the system about every six months.  The community is
working towards the next major release, aptly, \LaTeX{3}.

\section{The Strange Name}
\label{sec:strange-name}

The funny typesetting of the word \TeX{} is supposed to be a rendition
of the Greek letters $\tau \epsilon \chi$, which are transliterated as
\emph{tex}.  It should be pronounced ``tekh'' where ``kh'' indicates a
voiceless velar fricative (as a German would say ``J.S.\ Bach'').
However, most lazy Americans simply say ``tek.''  The name is in honor
of Caltech, where Donald Knuth did his graduate work.  A popular folk
etymology is that the name came from Knuth's verbal reaction to the
lousy typesetting of the second volume of \emph{The Art of Computer
  Programming}: ``blech.''

\LaTeX{} is for Lamport \TeX{}, after the initial author Leslie
Lamport.  It is usually pronounced ``lay-tek(h),'' although some say
``lah-tek(h).''  To say ``le-tek(h)'' is incorrect and should be
strictly shunned.  Apparently, the crazy typesetting Knuth used for
the word \TeX{} was contagiouis and was embraced by Lamport as the
word is printed as \LaTeX{}.

\section{The \LaTeX{} System}
\label{sec:latex-system}

As was painfully explained earlier, \LaTeX{} itself is a document
language.  More specifically, it is a macro language for the \TeX{}
language.  Just like all computer languages, it must either be
compiled or interpreted in order to be of any use.  The actual
document preparation takes place in a text editor, such as Emacs or
vi.  Vi is included in any POSIX-compliant operating system, such as
UNIX and most Unix-like operating systems.  Emacs is commonly found on
UNIX and Unix-like operating systems, and more recently, these editors
can be found on Mac OS X as well (which is really just bad UNIX).  You
can use DOS EDIT or Windows Notepad if you like wasting time.  A
usable document usually carries a \texext{} extension is an ordinary
ASCII text file.

In general terms, a compiler is a software program that translates a
human-readable source file into machine code that can be executed by a
computer.  Similarly, the \LaTeX{} compiler translates a \texext{}
file (the source) into a machine-readable format.  The output of the
standard \LaTeX{} compiler is what is called DVI for math for ``device
independent.''  A \dviext{} file will appear identical on any kind of
printer or computer display.  However, viewing a \dviext{} file on
anything other than a UNIX or Unix-like operating system proves to be
a chore as DVI viewers are less common on Mac OS and Microsoft
Windows.  Compilers have also been written to output a \texext{}
document directly into PostScript, PDF, or even HTML\@.  It is also
possible to convert from DVI to these formats.  PostScript is a
language common to most larger laserjet printers, and allows you to
print the document.  Additionally, PostScript was a common way to
exchange documents over the Internet prior to the advent of PDF files.
Adobe's Portable Document Format (PDF) is in some ways the successor
to PostScript files for exchange over the Internet.  They are readable
on nearly any piece of computer hardware that has a sufficiently-sized
screen, and enough memory to store the file.  It is important to note
that it is typically difficult to impossible to recover a \texext{}
document from a compiled document.  Thus, it is important to keep the
source files around if further editing is anticipated.

\LaTeX{} itself is fairly comprehensive and can accommodate a wide
variety of publishing needs.  It can, however, be extended with the
addition of user-defined macro packages.  A package is an independent
source file that adds features to the language by defining them in
terms of existing \LaTeX{} commands (these are called macros).  This
is the same principle by which \LaTeX{} is derived from Plain \TeX{};
\LaTeX{} is actually one big \TeX{} macro package.  An author can
include a package in a document and make use of these new features.
Packages can greatly simplify document creation by providing
additional commands that would take time and/or extensive programming
knowledge to implement.  The American Mathematical Society provides
macro packages that extend on the already rich mathematical
typesetting ability of \LaTeX{}.  This is collectively known as the
\AmS-\LaTeX{} system, and it can be considered to be the
mathematician's god package.  Also available are packages that add
support for new classes of documents, graphics, illustrations,
publishing in different languages, and even printing chess boards and
crossword puzzles.

\BibTeX{} is another important extension to \TeX{} and \LaTeX{}.
\BibTeX{} automates the creation of bibliographies and textual
citations.  A separate source file is used to store the publication
data for books, journals, and other articles that can be referred to
by a \TeX{} document.  An author can create a list of commonly used
citations and quickly refer to them in any document.  When compiled,
\BibTeX{} automates the formatting of citations and generates the
bibliography, all according to a customizable style package.  A large
number of \BibTeX{} packages are available to handle different
academic, legal, and professional citation styles.

Fonts for use with \LaTeX{} are usually written in \MF{} or \MP{} and
then compiled.  Since fonts are defined by geometric curves (the
curious can Google ``B\'ezier curve''), they can be extremely
versatile.  The same font can be used on any computer system for which
a \MF{} or \MP{} compiler is available.  Often, technical
illustrations are written in \MP{}.  Upon execution, the \LaTeX{}
compiler will reference the fonts that it needs and render them
appropriately.

The advantage of this approach is that \LaTeX{} is not married to any
specific operating platform.  A \texext{} document can be written on
any computer system with a text editor.  It is only necessary to
provide the compiler for a given operating system.  With a bit of
work, \LaTeX{} will run on anything that you can connect to a monitor,
keyboard, and printer to, or even those systems that you cannot
connect a printer to.  Furthermore, regardless of platform, \LaTeX{}
will produce identical output.

However, compiling and configuring a complete \LaTeX{} system is a
demanding and sometimes painful undertaking.  To remedy this, some
people have been kind enough to pre-package all of the necessary
software and automate the process.  These complete, ready-to-use
\LaTeX{} systems are called distributions.  \MiKTeX{} is the most
popular for Windows.  pro\TeX{t} is derived from \MiKTeX{} and can
ostensibly be run directly from a CD or flash drive.  te\TeX{} is a
common distribution on UNIX and Mac OS X, and \TeX{} Live is a
cross-platform collection of distributions.

While there are die hards who would rather face certain death than
stop using Emacs, and another group of die hards who await the day
when Emacs users will see the light and use vi, some people feel that
a text editor is not the most productive environment for producing
\LaTeX{} documents.  Furthermore, many people are not familiar with
Emacs or vi, and have no need to learn the intricacies of either
editor.  To this end, software developers have produced a number of
alternatives.  LyX and \TeX{macs} attempt to reproduce the
friendliness of Word or WordPerfect.  These programs are
pseudo-WYSIWYG (``what you see is what you get'') interfaces to an
underlying \LaTeX{} system.  However, this approach also obscures the
high degree of control the cryptic-looking source code provides.  The
\LaTeX{} front-end is another approach; a frontend is simply a text
editor with some creature comforts like syntax highlighting to help
you navigate your source code, toolbars for common commands, help with
matching parentheses, and project management.  These usually integrate
with an existing \LaTeX{} distribution and link to the \LaTeX{}
compiler; this allows you to compile your document without ever
leaving the front-end.  WinEdt and \TeX{nicCenter} are popular
front-ends for Windows, \TeX{Shop} is popular for Mac, and there are
many implementations for UNIX systems.  For those that still can't get
enough of their UNIX terminal, AUC\TeX{} adds additional \TeX{}
functionality to Emacs and \LaTeX{}-suite will extend the vim
implementation of vi with \LaTeX{}-friendly features.

\section{Documentation and the \TeX{} Community}
\label{sec:docum-tex-comm}

Because there are so many different components that make up a \LaTeX{}
system, documentation is unfortunately very scattered.  The
information that you need may not be part of \LaTeX{} itself, but a
specific package or perhaps your distribution.  The various
distributions are usually very good at providing the documentation to
the packages that they support.  However, there are so many packages
available that a problem might only be solved by an appeal to the
larger \TeX{} community.  The \href{http://www.ctan.org}{Comprehensive
  \TeX{} Archive Network (CTAN)} and the
\href{http://www.tug.org}{\TeX{} Users Group} are the two largest
Internet communities that provide repositories of \TeX{} packages and
documentation.  The benefit of the collaborative nature of \LaTeX{} is
its extreme versatility, but there is a price to be paid.  There is no
stupid cartoon paperclip to answer your questions.  Fortunately, the
answers are available to those willing to make a little effort.

%%% Local Variables: 
%%% mode: latex
%%% TeX-master: "reader"
%%% End: 
