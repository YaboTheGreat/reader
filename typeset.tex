\chapter{Typesetting Text}
\label{cha:typesetting-text}

\LaTeX{} differs from common word processors in that it requires the
author to indicate the logical structure of the document.  This
involves identifying sections, subsections, titles, and paragraphs
among others.  Only then will \LaTeX{} derive a text document that
looks remotely like what you intended it to be.  In this chapter, we
will cover the basic structure of every \LaTeX{} document in addition
to simple formatting commands available in \LaTeX{}.  We begin with
the syntactic structure of basic \LaTeX{} documents.

\section{Structure}
\label{sec:structure}

Like any programming language, \LaTeX{} enjoys a very strict and
unforgiving structure.

\subsection{Command Structure}
\label{sec:command-structure}

Commands in \LaTeX{} have a very intuitive and flexible structure,
like most programming languages.  Most of the commands that you will
encounter in \LaTeX{} will look like this:
\begin{verbatim}
\command{argument}
\end{verbatim}
where the name of the command replaces \texttt{command}, and any
argument or modifiers you have replace \texttt{argument}.  Some
examples include
\begin{verbatim}
\texttt{Text} \textit{Text} \textbf{Text}
\end{verbatim}
The first example will make text a cool
typewriter font, while the second and third examples will make your
text \textit{italic} and \textbf{bold}, respectively.  Note that all
commands start with a backslash.

\subsection{Environment Structure}
\label{sec:envir-struct}

Environments are special commands that modify large blocks of text.
Environments almost always start with \verb=\begin{environment}=, and
  end with \verb=\end{environment}=, where the name of the environment
replaces \textit{environment}.  The portion of the document between
the \verb=\begin{}= and \verb=\end{}= is in the selected environment.
An environment affects the behavior of \LaTeX{}.  For example, here is
an environment called \texttt{texttt} which alters the font \LaTeX{}
uses:
\begin{verbatim}
\begin{texttt}
  This is typewriter text
\end{texttt}
\end{verbatim}

Documents themselves are also environments:
\begin{verbatim}
\begin{document}
Document goes here.
\end{document}
\end{verbatim}

The second example is something that will be used quite often: it's
the beginning and ending arguments for a document!

\subsection{Document Structure}
\label{sec:document-structure}

Here's a sample document that outlines the basic structure of
\LaTeX{}:
\begin{boxedverbatim}
  \documentclass[12pt]{article}
  \usepackage{amsmath}
  
  \title{Some Really Important Results}
  \author{My Name}
    
  \begin{document}
  \maketitle
    
    \chapter{Chapter Name}
    A Document Body would go here.
    
    \section{The First Section}
    Here's the first section of my document.  It's
    wicked cool.
    
    \section{The Second Section}
    Here's the second section of my document.  This
    is twice as cool as the previous one.
    
    \subsection{A Subsection}
    Sometimes things aren't important enough to get
    their own section, so here's a subsection.
    
    \subsubsection{Subsubsections Exist?}
    I guess they do.
    
    \paragraph{Paragraph Title}
    Why paragraph have titles, I have no idea.
    
    \section*{The Third Section}
    This is the third section of my document, but
    this will show without a number because of the
    asterisk.
  \end{document}
\end{boxedverbatim}

When compiled, it'll look something like this:
\begin{framed}
  {\noindent\huge \textbf{Chapter 1}} \\[1cm]
  {\Huge \textbf{Chapter Name}} \\[0.25cm]
  A Document Body would go here. \\[0.5cm]
    
  {\noindent\Large \textbf{1.1 \quad The First Section}} \\[0.25cm]
  Here's the first section of my document.
  It's wicked cool. \\[0.5cm]
    
  {\noindent\Large \textbf{1.2 \quad The Second Section}} \\[0.25cm]
  Here's the second section of my document.
  This is twice as cool as the previous one. \\[0.5cm]
    
  {\noindent\large \textbf{1.2.1 \quad A Subsection}} \\[0.25cm]
  Sometimes things aren't important enough to get
  their own section, so here's a subsection. \\[0.5cm]
    
  \noindent\textbf{Subsubsections Exist?} \\[0.25cm]
  I guess they do. \\[0.5cm]
    
  \noindent\textbf{Paragraph Title} \quad Why paragraph have titles, I have no
  idea. \\[0.5cm]
    
  {\noindent\Large \textbf{The Third Section}} \\[0.25cm]
  This is the third section of my document, but this will show without
  a number because of the asterisk.
\end{framed}

\subsubsection{Preamble}
\label{sec:preamble}

One can see that before the actual document body, there's some weird
text at the beginning of our sample document.  This is called the
preamble of the document, and tells \LaTeX{} exactly how you want your
document to be structured.  Our sample document uses the
\texttt{article} document class, at 12pt font.  The title of your
document goes in between the braces in the command \verb=\title{}=,
while the author name(s) go in \verb=\author{}=.  You'll see that one
package has been declared with \verb=\usepackage{}=, \texttt{amsmath};
packages and \texttt{amsmath} will be covered in later chapters, so
we'll just ignore it for now.  The preamble is ended with the command
\verb=\begin{document}=, which begins our document.  One warning: In
order for our title to display in \LaTeX{}, the titling command
\verb=\maketitle= must be used \emph{immediately after}
\verb=\begin{document}= has been declared.

\subsection{Body}
\label{sec:body}

In most \LaTeX{} documents, you'll want to split your text up into
discrete sections or parts in order to make reading your document
accessible and easy to your readers.  \LaTeX{} makes this easy with
special commands that are placed at strategic places in your document.
To reiterate, Table~\ref{tab:sectioning-cmd} shows the available
sectioning commands.
\begin{table}[htbp]
  \centering
  \caption{Sectioning commands}
  \label{tab:sectioning-cmd}
  \begin{tabular}{@{}ll@{}}
    \toprule
    Command & Description \\
    \midrule
    \verb=\chapter{...}= & Starts a numbered chapter (Chapter 1) \\
    \verb=\section{...}= & Starts a numbered section (Section 1.1) \\
    \verb=\subsection{...}= & Starts a numbered subsection (Subsection
    1.1.1) \\
    \verb=\paragraph{...}= & Starts a paragraph (w/o numbering) \\
    \verb=\subparagraph{...}= & Starts a subparagraph (w/o numbering)
    \\
    \verb=\part{...}= & Starts a part (with Roman numeral
    numbering) \\
    \bottomrule
  \end{tabular}      
\end{table}

The mechanics of these commands will be explained later, but for now,
it'll suffice to say that each successive command creates a smaller
title, similar to the chapters and sections you see in your usual
textbook.  It should be noted that the \verb=\chapter= command is
available only in the \texttt{report} or \texttt{book} class.

Note: If you want any of the numbered commands without the numbers
(i.e., unmarked chapters), simply add a * at the end of the command.
For example, \verb=\section*{Section}= will yield: \\[0.25cm]
\begin{Large}
  Section
\end{Large} \\[0.25cm]
\noindent instead of \\[0.25cm]
\begin{Large}
  1.1 \quad Section
\end{Large}

\section{Formatting}
\label{sec:formatting}

Just like any other word processor, \LaTeX{} enjoys a good deal of
strict and unforgiving formatting commands.

\subsection{Emphasizing Words}
\label{sec:emphasizing-words}

Just like any word processor, you can \underline{underline} and
\textit{italicize} words whenever you see fit.  Underlining is done
with the command \verb=\underline{}=, where the underlined text
goes between the braces.  Italicization is done with
\verb=\textit{}=, with the desired italicized text between the
braces.  Likewise, \verb=\textbf{}= gives bold face.
\begin{framed}
\begin{verbatim}
I like to \underline{underline} text here and
\underline{everwhere}. I like it \textit{so}
much, I can't \textit{\underline{stop!}}
\end{verbatim}
  I like to \underline{underline} text here and
  \underline{everywhere}.  I like it \textit{so} much, I can't
  \textit{\underline{stop!}}
\end{framed}

\subsection{Left/Right/Center Justification}
\label{sec:leftr-just}

Justification in \LaTeX{} is easy to do.  If you want to left-align
text, you must put the text you want left-justified in the
\texttt{flushleft} environment.  That is between the commands
\verb=\begin{flushleft}= and \verb=\end{flushleft}=.
\begin{framed}
\begin{verbatim}
\begin{flushleft}
  Left-aligned text is fun.  I like left-aligned
  text. I think you should like left-aligned text
  too!
\end{flushleft}
\end{verbatim}
  \begin{flushleft}
    Left-aligned text is fun.  I like left-aligned text.  I think you
    should like left-aligned text too!
  \end{flushleft}
\end{framed}

If you have not figured it out already, right-aligned text is done
with the \texttt{flushright} environment.
\begin{framed}
\begin{verbatim}
\begin{flushright}
  Whoa, this text is right-aligned.  What the heck?
  Who uses right-aligned text anyway?
\end{flushright}
\end{verbatim}
  \begin{flushright}
    Whoa, this text is right-aligned.  What the heck?  Who uses
    right-aligned text anyway?
  \end{flushright}
\end{framed}

Centering text is also possible in \LaTeX{}, except instead of using
the \texttt{flushcenter} environment (which doesn't even exist), we
use the \texttt{center} environment by using the \verb=\begin{center}=
  and \verb=\end{center}= commands.
\begin{framed}
\begin{verbatim}
\begin{center}
  This text is centered.  Centered text makes me
  feel at peace. These examples make no sense,
  don't they?
\end{center}
\end{verbatim}
  \begin{center}
    This text is centered.  Centered text makes me feel at peace.
    These examples make no sense, don't they?
  \end{center}
\end{framed}
  
\subsection{Quotes and Verses}
\label{sec:quotes-verses}

When using quotation marks, it's probably a good idea \emph{not} to
use the \texttt{"} key for opening and closing quotes.  Rather, we use
two grave accent (` \!\!`) characters for opening quotes, and two
single quotes ('') for closing quotes.
\begin{description}
\item[Bad] I think it was Kant who said "Let my people go."  Or
  perhaps that was Moses.
\item[Good] I think it was Kant who said ``Let my people go.''  Or
  perhaps that was Moses.
\end{description}
If your quotation is long (say, more than four lines long), it might
be a good idea to use the \verb=\begin{quote}= and \verb=\end{quote}=
commands, as they will automatically be indented in text.  However,
separate paragraphs within the \texttt{quote} environment won't be
indented like regular paragraphs.  To do that, use the
\texttt{quotation} environment instead.  Furthermore, to typeset
poetic verses, use the \texttt{verse} environment.
\begin{framed}
  \noindent Quote:
\begin{verbatim}
One of my favorite quotes from \textit{Principles
of Mathematical Analysis} goes something like this:
\begin{quote}
  If f is a continuous complex function on [a,b],
  there exists a sequence of polynomials $P_n$ such
  that $\lim_{n\rightarrow \infty} P_n(x)=f(x)$
  uniformly on [a,b]. If f is real, the $P_n$ may
  be taken real.
\end{quote}

Math is very fun!
\end{verbatim}
  One of my favorite quotes from \textit{Principles of Mathematical
    Analysis} goes something like this:
  \begin{quote}
    If f is a continuous complex function on [a,b], there exists a
    sequence of polynomials $P_n$ such that $\lim_{n\rightarrow\infty}
    P_n(x)=f(x)$ uniformly on [a,b]. If f is real, the $P_n$ may be
    taken real.
  \end{quote}

  Math is very fun!
\end{framed}
\begin{framed}
  \noindent Quotation:
\begin{verbatim}
This text will contain a quotation with indentation
very soon.  Maybe about now!
\begin{quotation}
  This quotation has indentations at the beginning
  of every paragraph.

  The quick brown fox jumps over the lazy dog.
  Unfortunately, he jumped into hazardous terrain
  and died.

  Yoo hoo!
\end{quotation}
\end{verbatim}
  This text will contain a quotation with indentation very soon.
  Maybe about now!
  \begin{quotation}
    This quotation has indentations at the beginning of every
    paragraph.

    The quick brown fox jumps over the lazy dog.  Unfortunately, he
    jumped into hazardous terrain and died.

    Yoo hoo!
  \end{quotation}
\end{framed}
\begin{framed}
  \noindent Verse:
\begin{verbatim}
I know only one English poem by
heart. It is about Humpty Dumpty.
\begin{flushleft}
  \begin{verse}
    Humpty Dumpty sat on a wall:\\
    Humpty Dumpty had a great fall.\\
    All the King's horses and all
    the King's men\\
    Couldn't put Humpty together again.
  \end{verse}
\end{flushleft}
\end{verbatim}
  I know only one English poem by heart. It is about Humpty Dumpty.
  \begin{flushleft}
    \begin{verse}
      Humpty Dumpty sat on a wall:\\
      Humpty Dumpty had a great fall.\\
      All the King's horses and all
      the King's men\\
      Couldn't put Humpty together again.
    \end{verse}
  \end{flushleft}
\end{framed}

\subsection{Line and Page Breaks}
\label{sec:line-page-breaks}

\LaTeX{} automatically breaks up lines, adds spaces between words, and
auto-hyphenates words when necessary.  If you have noticed from this
reader, every line is optimized so the lengths are the same.
Normally, the first line of every paragraph is automatically indented,
with no additional space between paragraphs.  When it becomes
necessary to add a line break, two commands are available: \verb=\\=
and \verb=\newline=.  Both will break an additional line.  \LaTeX{}
will automaticallyl spill over to another page when there is too much
content for one page; hence, premature page breaks are done with the
command \verb=\newpage=.

If you would like to doublespace a document, this is not easily
accomplished with \LaTeX{}; however it is fairly straightforward with
the use of a package.  See Section~\ref{sec:line-spacing} in the
packages chapter.
    
\subsection{The Space Between Words\protect\footnote{Oetiker, Tobias, Hubert Partl, Irene Hyna, and Elisabeth Schlegl.  \emph{The Not So Short Introduction to \LaTeXe{}}.  May 2006.  p.\ 34}}
\label{sec:space-betw-words}

To get a straight right margin in the output, \LaTeX{} inserts varying
amounts of space between the words.  It inserts slightly more space at
the end of a sentence, as this makes the text more readable.  \LaTeX{}
assumes that sentences end with periods, question marks, or
exclamation marks.  If a period follows an uppercase letter, this is
not taken as a sentence ending, since periods after uppercase letters
normally occur in abbreviations.

Any exception from these assumptions has to be specified by the
author.  A backslash in front of a space generates a space that will
not be enlarged.  A tilde `$\sim$' character generates a space that
cannot be enlarged and additionally prohibits a line break.  The
command \verb=\@= in front of a period specifies that this period
terminates a sentence even when it follows an uppercase letter.

The additional space after periods can be disabled with the command
\verb=\frenchspacing=, which tells \LaTeX{} \emph{not} to insert more
space after a period than after ordinary characters.  This is very
common in non-English languages, except bibliographies.  If you use
\verb=\frenchspacing=, the command \verb=\@= is not necessary.

\section{The Title Block}
\label{sec:cover-page}

A title block is an easy way to make your document look professional;
it is very easy to do in \LaTeX{}.  A title block in \LaTeX{} contains
the title of the document, the author(s), and optionally the date. You
can place a title block either at the top of your document or within a
cover page, which is on a separate page and is not numbered; however,
the cover page is only natively available in the \texttt{report} and
\texttt{book} classes (you have to specify the \texttt{titlepage}
option as an optional argument to the \texttt{article} document class
to get it).

To make a title page (or a title block), you need to declare its
contents by including
\begin{verbatim}
\title{Why Word Sucks}
\author{The Berkeley \LaTeX{} DeCal Staff
\thanks{Supported by Professor Ribet} \\ Berkeley, CA
\and The \LaTeX{} Community \\ Around the World}
\date{2005}
\end{verbatim}
in the preamble, or any place before the command to make the cover
page given by \verb=\maketitle=.  The \verb=\maketitle= command should
then be placed right after the \verb=\begin{document}= command.

As you can see in the sample declaration of the contents of the
title block, you can use the \verb=\and= command to introduce
multiple authors.  The \verb=\thanks= command creates footnotes
(automatically numbered, of course; would you expect anything less
from \LaTeX{}?) at the bottom of the cover page.  This can be
accomplished by using the \verb=\\= command.  If you do not use the
\verb=\date= command, today's date will automatically be inputted.

\section{References}
\label{sec:references}

The way that \LaTeX{} deals with cross-references to other sections,
tables, figures, or theorems is one of the best attributes of
\LaTeX{}.  Usually, when referencing different sections, tables,
figures, pages, etc., you have to manually look up what the number is
and place it in your document.  This is fine if you never make any
changes to your document, but we all know that everything seems to be
revised constantly.  This can make it a huge pain to go back through
all of your references to change all of them by one number.  \LaTeX{}
avoids this annoyance by having a way to define and recall references.

In order to reference something in \LaTeX{}, you need to define a
label for the reference.  This is done by the command
\verb=\label{marker}= where \texttt{marker} is the reference that you
will need to remember to reference the part later.  The \verb=\label=
command saves the last number that was generated so it is usually best
to put it in the same line as the \verb=\begin= command for that part
to make things standardized.

If you are cross-referencing a table or a figure, you will need to put
the label in the \verb=\caption= command.  Here is an example below:
\begin{verbatim}
\begin{figure}[!h]
  \includegraphics{filename}
  \caption{I hate graphics in \LaTeX{}}
  \label{fig:hategraphics}
\end{figure}
\end{verbatim}

Now if you want to reference the number of the thing you labeled, use
the \verb=\ref{marker}= command using the \texttt{marker} that you
chose before.  If I wanted to reference the figure example above, I
would type
\begin{verbatim}
\ref{fig:hategraphics}
\end{verbatim}
You can also use \verb=~\ref{marker}= without a space between the word
that precedes it.  To use this with the figure example above, I would
type
\begin{verbatim}
Figure~\ref{fig:hategraphics}
\end{verbatim}
You can also reference the page number of a label by using the
\texttt{pageref} command that is used in the same way as the
\texttt{ref} command.

\section{Footnotes}
\label{sec:footnotes}

In order to create footnotes in a \LaTeX{} document, you use the
\verb=\footnote= command.  So that you know you are using
footnotes correctly, footnotes should be placed after the word or
sentence (after the comma or period) they are referring to. In the
event a footnote within a section command is required, precede the
\verb|\footnote| command with \verb|\protect|, like so:
\begin{verbatim}
\section{Foo\protect\footnote{This is a footnote!}}
\end{verbatim}
Here is an example of a footnote in action:
\begin{verbatim}
I like bananas. I know that mangoes are sweet. I like
papayas. But I know that nothing can beat\ldots{}
\LaTeX{}! Hell yes!\footnote{Based on a chant from %
World Youth Day.}
\end{verbatim}
I like bananas. I know that mangoes are sweet. I like papayas. But I
know that nothing can beat\ldots{}\LaTeX{}!  Hell yes!\footnote{Based on a
  chant from World Youth Day.}

%%% Local Variables: 
%%% mode: latex
%%% TeX-master: "reader"
%%% End: 