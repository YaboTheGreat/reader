\chapter{Software}
\label{cha:software}

Nearly all the software you need is freely distributed, with some
frontends being the notable exception.  As a consequence, \LaTeX{}
runs quite well on just about any modern platform.

The software necessary for using \LaTeX{} mostly falls into three
categories: \LaTeX{} distributions, editors, and frontends.  \LaTeX{}
distributions include the \LaTeX{} compiler, and generally come with a
DVI viewer, and some other useful programs for working with \LaTeX{}
documents.

Editors allow you to manipulate plain text files, such as \LaTeX{}
source files.  A basic editor is included with most modern operating
systems.  For the Mac, SimpleText was the included basic editor for a
very long time, and more recent versions of the Mac OS (Mac OS X)
include a more sophisticated editor, TextEdit.  Windows offers
Notepad, which is a bit limited, as well as WordPad in all current
versions.  Lastly, Unix operating systems typically have many
editors.  These include vi, its cousin vim, Emacs, Pico, and Pico's
GNU counterpart Nano.

Even the best editor is still focused on the task of editing text
files.  While a text editor is the most basic requirement to use
\LaTeX{}, there is another class of programs called frontends, which
seek to simplify the task of creating \LaTeX{} documents.  Frontends
typically include editor capabilities that are specifically geared
towards \LaTeX{}.  This means that source code is often coded for
clarity, the pairing of delimiters is sometimes tracked, and some have
auto-complete functions based on what is currently being typed.  The
most important feature in most frontends, however, is some degree of
integration with a compiler.  This allows you to edit, compile, and
view your document all from within a single program.  Some frontends
even go so far as to let you view the compiled code in real-time, or
near real-time, and edit the underlying source code graphically from
the compiled version.

\section{Installing a \LaTeX{} Distribution}
\label{sec:inst-latex-distr}

The first and most important thing to have before creating \LaTeX{}
documents is a \LaTeX{} distribution.  This section will describe some
of the most common \LaTeX{} distributions for different operating
systems.

\subsection{Windows 95/98/98SE/ME/2000/XP/Vista}
\label{sec:wind-959898s}

\href{http://www.miktex.org}{\MiKTeX{}} is a very popular \LaTeX{}
distribution for Windows.  Before installing, though, you have decide
whether to go for a barebones \MiKTeX{} installation and then download
the packages as you need them or a complete installation of \MiKTeX{}
with all packages and other optional files installed.  Regardless of
the type of installation you want, to install it, download the
appropriate installer.  Once the download is finished, run the
installer.  It will then connect to the Internet and download the
files needed to install \MiKTeX{}.  After the installer finishes
downloading everything, close the installer and run it again.  This
time, let it install the distribution from a local directory (the
temporary directory the installer saved all the files to).  After a
few minutes, \MiKTeX{} should be installed.

\subsection{Mac OS X and Unix}
\label{sec:mac-os-x}

\TeX{} Live is the recommended \LaTeX{} distribution for Mac OS X and
Unix.  You can go to their web site at
\url{http://www.tug.org/texlive/} for more information and to download
the installer.

\section{Frontends}
\label{sec:frontends}

At this point, you have a working \LaTeX{} installation.  While you
could use a regular text editor to make your \LaTeX{} files (discussed
in the next section), using a dedicated \LaTeX{} frontend is
recommended.  Here are a couple of \LaTeX{} frontends for Windows and
Macs.

\subsection{Windows}
\label{sec:windows}

\TeX{}nicCenter is an open source and freely distributable frontend to
\LaTeX{}.  It integrates the \LaTeX{} and \BibTeX{} compilers as well
as a menu-driven interface to insert common \LaTeX{} commands.  You
can download \TeX{}nicCenter at \url{http://www.toolscenter.org/}.

\subsection{Mac OS X}
\label{sec:mac-os-x-1}

\TeX{}Shop is an open source and freely distributable frontend for
\LaTeX{} on the Mac OS X.  Like \TeX{}nicCenter for Windows, it
integrates the \LaTeX{} compilers together and offers a user-friendly
interface for \LaTeX{} beginners.  You can download \TeX{}Shop at
\url{http://www.uoregon.edu/~koch/texshop/}.

\subsection{Multiplatform}
\label{sec:multiplatform}

AUC\TeX{} is a multiplatform \LaTeX{} frontend that interfaces with
Emacs to provide a comprehensive editing environment for \TeX{}
documents.  It has many features that are too numerous to mention in
here, but you can find out more at
\url{http://www.gnu.org/software/auctex/}.

\section{Text Editors}
\label{sec:text-editors}

While using a frontend is recommended, there are times when a simple
text editor is the best option (and in some cases, the only option).
Some examples that have been mentioned previously are Emacs, vi (and
its cousin vim), Pico, and Nano.

%%% Local Variables: 
%%% mode: latex
%%% TeX-master: "reader"
%%% End: 