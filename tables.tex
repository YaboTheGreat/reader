\chapter{Tables}
\label{cha:tables}

Matrices are really wonderful if you want to make, well, matrices.
But what if you wanted to put words in your matrix?  Well, if you
really wanted, you could probably use \texttt{mathrm} and make a giant
ugly mess.  But in reality, your scary mutant matrix would be a
table.  To avoid this untimely end to your sanity, the
\texttt{tabular} environment exists.  Much like matrices, ampersands
are used to separate entries in a row and double backslashes are used
to end lines.  Additionally, one can control the justification of each
of the individual columns and add vertical lines between the columns.
In order to avoid continuing to describe tables in hazy terms, we are
just going to cut to the chase and have an example.
\begin{verbatim}
\begin{tabular}{|c|c|}
  a & b \\
  c & d \\
\end{tabular}
\end{verbatim}
makes the following table:
\begin{center}
\begin{tabular}{|c|c|}
  a & b \\
  c & d \\
\end{tabular}
\end{center}
Now that was cool, but you've probably noticed that this example table
has no horizontal lines, and as a result, is rather ugly.  That's
where \verb|\hline| comes in.  \verb|\hline| draws a horizontal line.
So if you want horizontal lines in the table above, you'd change
things to:
\begin{verbatim}
\begin{tabular}{|c|c|}
  \hline
  a & b \\
  \hline
  c & d \\
  \hline
\end{tabular}
\end{verbatim}
to get:
\begin{center}
\begin{tabular}{|c|c|}
  \hline
  a & b \\
  \hline
  c & d \\
  \hline
\end{tabular}
\end{center}
Note that you need a line break before the last \verb|\hline| (if you
didn't use \verb|\hline|, you wouldn't need a line break after the
last set of entries).  So now that we have a pretty table, how did we
get there?

\section{Anatomy of the \texttt{tabular} Environment}
\label{sec:anat-tabular-envir}

To start, we see that things are laid out similarly to a matrix, with
each cell in a row separated by an ampersand and rows being ended with
line breaks.  Notice that we put some arguments after the
\verb|\begin{tabular}|.  The vertical bars, |, draw the lines down
each column.  The \texttt{c}'s center the contents of each cell.
The other options are \texttt{l} and \texttt{r} for left and right
justification, respectively.  But what if you don't want vertical
lines?  You simply leave out the bars and continue as usual.  By
default, entries are centered vertically within the cell, but, not
surprisingly, you have the option of changing this to either the top,
\texttt{t}, or bottom, \texttt{b}.  This option, should you desire it,
goes in square brackets immediately following the
\verb|\begin{tabular}|.  So, if we wanted a table with the entries at
the bottom of the cells, we'd have:
\begin{verbatim}
\begin{tabular}[b]{|c|c|}
  \hline
  a & b \\
  \hline
  c & d \\
  \hline
\end{tabular}
\end{verbatim}
which produces:
\begin{center}
\begin{tabular}[b]{|c|c|}
  \hline
  a & b \\
  \hline
  c & d \\
  \hline
\end{tabular}
\end{center}

\subsection{Fiddling with Columns}
\label{sec:fiddl-with-columns}

\LaTeX{} automatically decides how wide to make each column, but, as
with many things in \LaTeX{}, you can override it, should you need to
do so.  Likewise, you can make a single cell span more than one
column. To set the width of a column manually, you change the
alignment.  Instead of \texttt{l}, \texttt{r}, or \texttt{c}, you use
\verb|p{}| and put the width of the column inside the curly braces.
\verb|\multicolumn| and \verb|\cline| are two other commands that
override the layout of a table.  \verb|\multicolumn| lets you have a
cell span that is more than one column.  It takes three arguments: the
number of columns spanned, the alignment within the cell, and the
actual contents.  \verb|\cline{a-b}| is a variation on \verb|\hline|
which draws a horizontal line from column \texttt{a} to column
\texttt{b}.  There are other far more complicated things that can be
done with table alignment, but you'll have to look in a bigger, far
more complicated, book for that.

\section{The \texttt{table} Environment and Captions}
\label{sec:table-envir-capt}

It would look awfully silly if tables were split in two at page
breaks.  As you might imagine, tables don't do that.  If a table falls
on a page break, \LaTeX{} will move it entirely onto the next page.
But what if you want to caption your table?  You'd want the caption to
ride along with the table over the page breaks, wouldn't you?  Of
course, you could always align the caption yourself, but why do extra
work when you could use the \texttt{table} environment?
\texttt{table} doesn't replace \texttt{tabular}; rather, it encloses
\texttt{tabular}.  Within the \texttt{table} environment, you can use
\verb|\caption| to caption your table (either above or below the
\texttt{tabular} environment).  Additionally, the \verb|\begin{table}|
command takes an optional argument that controls the placement of the
table.  The options are \texttt{h} to place the table where it appears
in the source, \texttt{t} to place the table at the top of a page,
\texttt{b} to place the table at the bottom of a page, and \texttt{p}
to place the page on its own page of floats.

\section{Example}
\label{sec:example}

Up to this point, we've had fairly silly, uninteresting examples.
This section is just going to be one big example.  To render the table
exactly as typed, the \texttt{booktabs} and \texttt{graphicx} packages
must be loaded.
\begin{table}
  \centering
  \caption{Career statistics: Brant Brown}
  \label{tab:brant-brown}
  \resizebox{\textwidth}{!}{\begin{tabular}{@{}rrrrrrrrrrrrrrrrr@{}}
    \toprule
    Year & Tm & G & AB & R & H & 2B & 3B & HR & RBI & SB & CS & BB &
    SO & BA & OBP & SLG \\
    \midrule
    1996 & CHC & 29 & 69 & 11 & 21 & 1 & 0 & 5 & 9 & 3 & 3 & 2 & 17 &
    .304 & .329 & .536 \\
    1997 & CHC & 46 & 137 & 15 & 32 & 7 & 1 & 5 & 15 & 2 & 1 & 7 & 28
    & .234 & .286 & .409 \\
    1998 & CHC & 124 & 347 & 56 & 101 & 17 & 7 & 14 & 48 & 4 & 5 & 30
    & 95 & .291 & .348 & .501 \\
    1999 & PIT & 130 & 341 & 49 & 79 & 20 & 3 & 16 & 58 & 3 & 4 & 22 &
    114 & .232 & .238 & .449 \\
    2000 & FLA & 41 & 73 & 4 & 14 & 6 & 0 & 2 & 6 & 1 & 0 & 3 & 33 &
    .192 & .224 & .356 \\
    \multicolumn{2}{r}{CHC} & 54 & 89 & 7 & 14 & 1 & 0 & 3 & 10 & 2 &
    1 & 10 & 29 & .157 & .248 & .270 \\
    \multicolumn{2}{r}{TOT} & 95 & 162 & 11 & 28 & 7 & 0 & 5 & 16 & 3
    & 1 & 13 & 62 & .173 & .237 & .309 \\
    \addlinespace
    \multicolumn{2}{l}{5 Seasons} & 424 & 1056 & 142 & 261 & 52 & 11 &
    45 & 146 & 15 & 14 & 74 & 316 & .247 & .301 & .445 \\
    \bottomrule
  \end{tabular}}
\end{table}

\begin{verbatim}
\begin{table}
  \centering
  \caption{Career Statistics: Brant Brown}
  \label{tab:brant-brown}
  \resizebox{\textwidth}{!}{\begin{tabular}{@{}rrrr %
        rrrrrrrrrrrrr@{}}
    \toprule
    Year & Tm & G & AB & R & H & 2B & 3B & HR & RBI &
    SB & CS & BB & SO & BA & OBP & SLG \\
    \midrule
    1996 & CHC & 29 & 69 & 11 & 21 & 1 & 0 & 5 & 9 &
    3 & 3 & 2 & 17 & .304 & .329 & .536 \\
    1997 & CHC & 46 & 137 & 15 & 32 & 7 & 1 & 5 & 15
    & 2 & 1 & 7 & 28 & .234 & .286 & .409 \\
    1998 & CHC & 124 & 347 & 56 & 101 & 17 & 7 & 14 &
    48 & 4 & 5 & 30 & 95 & .291 & .348 & .501 \\
    1999 & PIT & 130 & 341 & 49 & 79 & 20 & 3 & 16 &
    58 & 3 & 4 & 22 & 114 & .232 & .238 & .449 \\
    2000 & FLA & 41 & 73 & 4 & 14 & 6 & 0 & 2 & 6 & 1
    & 0 & 3 & 33 & .192 & .224 & .356 \\
    \multicolumn{2}{r}{CHC} & 54 & 89 & 7 & 14 & 1 &
    0 & 3 & 10 & 2 & 1 & 10 & 29 & .157 & .248 & .270
    \\
    \multicolumn{2}{r}{TOT} & 95 & 162 & 11 & 28 & 7
    & 0 & 5 & 16 & 3 & 1 & 13 & 62 & .173 & .237 &
    .309 \\
    \addlinespace
    \multicolumn{2}{l}{5 Seasons} & 424 & 1056 & 142
    & 261 & 52 & 11 & 45 & 146 & 15 & 14 & 74 & 316 &
    .247 & .301 & .445 \\
    \bottomrule
  \end{tabular}}
\end{table}
\end{verbatim}

%%% Local Variables: 
%%% mode: latex
%%% TeX-master: "reader"
%%% End: 
